\documentclass[10pt]{article}

\usepackage{latex2anki}

% some user macros:
\newcommand{\CC}{\mathbb{C}}
\newcommand{\RR}{\mathbb{R}}
\newcommand{\ZZ}{\mathbb{Z}}

% use \def instead of \renewcommand
\def\phi{\varphi}

% DeclareMathOperator should also work
\DeclareMathOperator{\Mat}{Mat}
   
\begin{document}

\begin{note}{mydeck-0001}
  \field Title of Card 1
  \field
  This is a \cloze{1}cloze\clend\
  deletion test.  You can have several \cloze{2}clozes\clend\ on the same card.
  In the pdf preview, they get displayed in different colours.

  You can also define a \cloze{3}cloze with a hint\hint hint\clend.
  \field
  You can display some additional info on the back of the card.
  \field Test Chapter
  \field Example 
  \field 01
\end{note}
 
\begin{note}{mydeck-0002}
  \field Preimage
  \field
  A \textbf{preimage} of an element \(b \in B\) under a map \(f\colon A\to B\) is \cloze{1}an element \(a\in A\) with \(f(a) = b\).\hint (think of an element in \(A\))\clend    
  \field  
  \field Set Theory
  \field definition
  \field 05
\end{note}
 
\begin{note}{mydeck-0003}
  \field
  User macros
  \field
  User \cloze{1}marcos\clend\ get expanded by plastex: \(\RR\), \(\CC\)
  \field
  \field Test Chapter
  \field Example
  \field 02
\end{note}

\begin{note}{mydeck-0004}
  \field
  Tables
  \field
  LaTeX tabulars get converted to html tables:
  \cloze{1}
  \begin{center}
    \begin{tabular}{ll}
      \(a\) & \(b\) \\
      \(c\) & \(d\)
    \end{tabular}
  \end{center}
  \clend
  \field
  \field Test Chapter
  \field Example
  \field 02
\end{note}

\begin{note}{mydeck-0005}
  \field
  Aligned equations
  \field
  Aligned equations that contain cloze deletions are best simulated using tabulars:
  \begin{center}
    \begin{tabular}{rcl}
      & & \cloze{1} \(f(\vec v_1, \dots, \vec v_{i-1}, \vec v_i + \vec v_i' , \vec v_{i+1},\dots, \vec v_n)\) \clend \\
      & & \(=\) \cloze{2} \(\begin{aligned}
                          & f(\vec v_1, \dots, \vec v_{i-1}, \quad\vec v_i\quad, \vec v_{i+1},\dots, \vec v_n)
                          \\
                          +\,
                          & f(\vec v_1, \dots, \vec v_{i-1}, \quad\vec v_i'\quad , \vec v_{i+1},\dots, \vec v_n)
                        \end{aligned}\) \clend \\
      ~\\
      & & \cloze{3} \(f(\vec v_1, \dots, \vec v_{i-1}, s\vec v_i , \vec v_{i+1},\dots, \vec v_n)\) \clend \\
      & = & \cloze{4} \(s \cdot f(\vec v_1, \dots, \vec v_{i-1}, \phantom{s} \vec v_i, \vec v_{i+1},\dots, \vec v_n)\) \clend
    \end{tabular}
  \end{center}
  \field
  \field Test Chapter
  \field Example
  \field 03
\end{note}


\end{document}


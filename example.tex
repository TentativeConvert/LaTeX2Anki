\documentclass[10pt]{article}

\usepackage{latex2anki}
%%%%%%%%%%%%%%%%%%%%%%%%%%%%%%%%%%%%%%%%%%%%%%%%%%
% The following should also be included in anki.sty, but currently that breaks tikz-cd.
% See https://github.com/plastex/plastex/issues/407#issue-3375113580
\newif\ifplastex
% 
\ifplastex\else
  \definecolor{mygray}{gray}{0.96}
  \RequirePackage[a4paper]{geometry}
  \geometry{paperwidth=.5\paperwidth,paperheight=.25\paperheight,left=2em,right=2em,bottom=1em,top=2em}
  \RequirePackage{pgfpages}
  \pgfpagesuselayout{8 on 1}[a4paper,border shrink=1mm]
  % TODO: bring back page borders
  % see https://tikz.dev/pages#pgf.back/pgfpagesuselayout
  % but
  %     \pgfpageslogicalpageoptions{1}{border code=\pgfstroke}
  %     does not compile
  % Current work-around is to set logical page color to light gray
  % and leave physical page colour undefined (white);
  % Would prefer to set physical page color to gray
  % and logical page color to white, but for some reason then everything becomes white again.
  \pagestyle{empty}
  \setlength{\parindent}{0in}
\fi
%%%%%%%%%%%%%%%%%%%%%%%%%%%%%%%%%%%%%%%%%%%%%%%%%%

% blackboard bold font letters
\newcommand{\CC}{\mathbb{C}}
\newcommand{\FF}{\mathbb{F}}
\newcommand{\NN}{\mathbb{N}}
\newcommand{\QQ}{\mathbb{Q}}
\newcommand{\RR}{\mathbb{R}}
\newcommand{\ZZ}{\mathbb{Z}}
%
% (redefined) symobls 
%
\newcommand{\abs}[1]{\left\vert#1\right\vert}
\newcommand{\norm}[1]{\left\|#1\right\|}
\newcommand{\scprod}[2]{\langle #1, #2\rangle} %Skalarprodukt
\def\phi{\varphi}
\def\epsilon{\varepsilon}
\def\leq{\leqslant}
\def\geq{\geqslant}
\newcommand{\tsmash}{\wedge}
\newcommand{\twedge}{\vee}
\newcommand{\from}{\leftarrow}
\newcommand{\noloc}{:\!}
%
% special maps
%
\newcommand{\id}{\mathrm{id}}
%
% linear algebra
%
\def\vec #1{\mathbf{#1}}
\newcommand{\Potenz}{\mathfrak P}
\newcommand{\hull}[1]{\langle #1 \rangle}
\newcommand{\Lraum}{\mathcal L}

% Categories:
\DeclareMathOperator{\Ar}{Ar}
\newcommand{\cat}[1]{\bm{\mathsf{#1}}}
\newcommand{\iTop}{\underline{\smash{\mathsf{Top}}}}  % internal hom in Top
\newcommand{\iAut}{\underline{\smash{\mathsf{Aut}}}}  
\newcommand{\Aut}{\mathrm{Aut}}
\newcommand{\Mor}{\mathrm{Mor}}
\newcommand{\Hom}{\mathrm{Hom}}
\DeclareMathOperator{\signum}{sgn}
\DeclareMathOperator{\rank}{rk}
\DeclareMathOperator{\im}{im}
\DeclareMathOperator{\Abb}{Abb}
\DeclareMathOperator{\Mat}{Mat}


%%%%%%%%%%%%%%%%%%%%%%%%%%%%%%%%%%%%%%%%%%%%%%%%%%
% additional commands in LinA2022.tex:
% 
% 
%\mathchardef\mhyphen="2D
%\newcommand*{\factor}[2]{\left.\raisebox{.1em}{\ensuremath{#1}}\middle/\raisebox{-.1em}{\ensuremath{#2}}\right.}
% 
%% arrows:
%\newcommand*{\openhookrightarrow}{\mathrel{\ooalign{$\hookrightarrow$\cr\hidewidth\hbox{$\circ\,\,$}\cr}}}
%\newcommand*{\closedhookrightarrow}{\mathrel{\ooalign{$\hookrightarrow$\cr\hidewidth\raise0.1ex\hbox{$\rule{0.5pt}{1ex}\;\;$}\cr}}}
%\newcommand*{\openhookleftarrow}{\mathrel{\ooalign{$\hookleftarrow$\cr\hidewidth\hbox{$\circ\;$}\cr}}}
%\newcommand*{\closedhookleftarrow}{\mathrel{\ooalign{$\hookleftarrow$\cr\hidewidth\raise0.1ex\hbox{$\rule{0.5pt}{1ex}\,\;$}\cr}}}
%\newcommand*{\leadsleadsto}{\mathrel{\substack{\leadsto\\[-1em]\leadsto}}}
% 
% 
%% text positioning in maths:
%\newcommand{\ctext}[2]{\text{\parbox{#1}{\centering #2}}}
%\newcommand{\ltext}[2]{\text{\parbox{#1}{\raggedright#2}}}
%\newcommand{\rtext}[2]{\text{\parbox{#1}{\raggedleft#2}}}
%\newcommand{\cttext}[2]{\text{\parbox[t]{#1}{\centering #2}}}
%\newcommand{\lttext}[2]{\text{\parbox[t]{#1}{\raggedright#2}}}
%\newcommand{\rttext}[2]{\text{\parbox[t]{#1}{\raggedleft#2}}}
%%%%%%%%%%%%%%%%%%%%%%%%%%%%%%%%%%%%%%%%%%%%%%%%%%

   
\begin{document}

\begin{note}{f825555f-986c-4044-b28b-5af09385be21}
  \field De Morgansche Regel (I)
  \field
  Das Komplement einer Vereinigung ist \cloze{1}der Schnitt der Komplemente.\clend
  \field
  \field Mengen
  \field Satz
  \field 01.08
\end{note}
 
\begin{note}{ad21153d-ddfa-4d80-9843-ace7b8f615c8}
  \field De Morgansche Regel (II)
  \field
    Das Komplement eines Schnitts ist \cloze{1}die Vereinigung der Komplemente.\clend
  \field  
  \field Mengen
  \field Satz
  \field 01.08
\end{note}
 
\begin{note}{11291ec1-e99f-4642-b7c1-7581a2a8e902}
  \field
  \field
  Ein \textbf{Urbild} eines Elements \(b\in B\) unter einer Abbildung \(f\colon A\to B\) ist \cloze{1}ein Element \(a\in A\) mit \(f(a) = b\).\hint Element\clend
  \field
  \field Mengen
  \field Definition
  \field 01.17
\end{note}
 
 
\begin{note}{d9ba908c-fc60-4836-88c1-4810fb0a578d}
  \field
  \field Das \textbf{Urbild} einer Teilmenge \(B'\subset B\) unter einer Abbildung \(f\colon A\to B\) ist \cloze{1}die Menge \emph{aller} \(a\in A\) mit \(f(a) \in B'\).\clend
  \field
  \field Mengen
  \field Definition
  \field 01.17
\end{note}
 
\begin{note}{547c84db-12f0-4c3c-afe4-1d28b22ad5fa}
  \field
  \field Die \textbf{Faser} eines Elements \(b\in B\) bezüglich einer Abbildung \(f\colon A\to B\) ist \cloze{1}die Menge aller Urbilder von \(b\).\clend
  \field
  \field Mengen
  \field Definition
  \field 01.17
\end{note}

\begin{note}{TESTmatrix}
  \field
  Zeilennormalform
  \field
  Eine Matrix in \textbf{Zeilennormalform} (ZNF) ist \cloze{1} eine Matrix in Zeilenstufenform, in der alle Pivot-Element \(1\) sind und oberhalb jedes Pivot-Elements nur Nullen stehen.\clend
  \field
  \[
    \begin{pmatrix}
      0 & 0 & \textcolor{red}{1} & \blacksquare & 0 & \blacksquare & \blacksquare & 0 & \blacksquare & \blacksquare  \\
      0 & 0 & 0 & 0            & \textcolor{red}{1} & \blacksquare & \blacksquare & 0 & \blacksquare & \blacksquare  \\
      0 & 0 & 0 & 0            & 0 & 0            & 0            & \textcolor{red}{1} & \blacksquare & \blacksquare  \\
      0 & 0 & 0 & 0            & 0 & 0            & 0            & 0 & 0            & 0             
    \end{pmatrix}
  \]
  \field
  \field
  \field
\end{note}

\begin{note}{TEST}
  \field
  aligned (mit internen clozes)
  \field
  \(V\), \(W\) Vektorräume. Eine Abbildung \(f\colon V^n\to W\) ist multilinear, wenn gilt:
  \[
  \begin{aligned} & & & \cloze{1} f(\vec v_1, \dots, \vec v_{i-1}, \vec v_i + \vec v_i' , \vec v_{i+1},\dots, \vec v_n) \clend \\
    = & \phantom{\,+\,} & & \cloze{2} f(\vec v_1, \dots, \vec v_{i-1}, \quad\vec v_i\quad, \vec v_{i+1},\dots, \vec v_n) \\
                  & \,+\, & & f(\vec v_1, \dots, \vec v_{i-1}, \quad\vec v_i'\quad , \vec v_{i+1},\dots, \vec v_n) \clend \\
    ~\\
                  & & & \cloze{3} f(\vec v_1, \dots, \vec v_{i-1}, s\vec v_i , \vec v_{i+1},\dots, \vec v_n) \clend \\
    = & & & \cloze{4} s \cdot f(\vec v_1, \dots, \vec v_{i-1}, \phantom{s} \vec v_i, \vec v_{i+1},\dots, \vec v_n) \clend
  \end{aligned}
  \]
  \field
  \field Determinante
  \field Definition
  \field 08.04
\end{note}

\begin{note}{TESTtable1}
  \field
  Tabelle (tabular)
  \field
  \begin{center}
    \begin{tabular}{ll}
      \(a\) & \(b\) \\
      \(c\) & \(d\)
    \end{tabular}
  \end{center}
  \field
  \field
  \field
  \field
\end{note}

\begin{note}{TESTtable2}
  \field
  Simulation von aligned mit tabular
  \field
  \(V\), \(W\) Vektorräume. Eine Abbildung \(f\colon V^n\to W\) ist multilinear, wenn gilt:
  \begin{center}
    \begin{tabular}{rcl}
      & & \cloze{1} \(f(\vec v_1, \dots, \vec v_{i-1}, \vec v_i + \vec v_i' , \vec v_{i+1},\dots, \vec v_n)\) \clend \\
      & & \(=\) \cloze{2} \(\begin{aligned}
                          & f(\vec v_1, \dots, \vec v_{i-1}, \quad\vec v_i\quad, \vec v_{i+1},\dots, \vec v_n)
                          \\
                          +\,
                          & f(\vec v_1, \dots, \vec v_{i-1}, \quad\vec v_i'\quad , \vec v_{i+1},\dots, \vec v_n)
                        \end{aligned}\) \clend \\
      ~\\
      & & \cloze{3} \(f(\vec v_1, \dots, \vec v_{i-1}, s\vec v_i , \vec v_{i+1},\dots, \vec v_n)\) \clend \\
      & = & \cloze{4} \(s \cdot f(\vec v_1, \dots, \vec v_{i-1}, \phantom{s} \vec v_i, \vec v_{i+1},\dots, \vec v_n)\) \clend
    \end{tabular}
  \end{center}
  \field
  \field Determinante
  \field Definition
  \field 08.04
\end{note}


\begin{note}{tikz1}
  \field
  TIKZ1
  \field
  \cloze{1}ABCD\clend
  \begin{center}
    \begin{tikzcd}[row sep=1cm]
      A \arrow{r}{a} \arrow{d}{b} & B \arrow{d}{c}\\
      C \arrow{r}{d} & D
    \end{tikzcd}
  \end{center}
  \field
  \field
  \field
  \field
\end{note}

\begin{note}{tikz2}
  \field
  TIKZ2
  \field
  \cloze{1}EF\clend
  \begin{center}
    \begin{tikzcd}[row sep=1cm]
      E \arrow{r}{a}  & F
    \end{tikzcd}
  \end{center}
  \field
  \field
  \field
  \field
\end{note}

\end{document}

